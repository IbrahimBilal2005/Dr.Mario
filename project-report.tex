\documentclass{article}

%% Page Margins %%
\usepackage{geometry}
\geometry{
    top = 0.75in,
    bottom = 0.75in,
    right = 0.75in,
    left = 0.75in,
}

\usepackage{amsmath}
\usepackage{graphicx}
\usepackage{parskip}

\title{Assembly Project: Dr Mario}

% TODO: Enter your name
\author{Ahnaf Keenan Ardhito & Ibrahim Bilal}

\begin{document}
\maketitle

\section{Instruction and Summary}

\begin{enumerate}

    \item Which milestones were implemented? 
    \begin{enumerate}
        \item Milestone 1
        \item Milestone 2
        \item Milestone 3
        \item Milestone 4 and 5 :
        \begin{enumerate}
            \item Easy:
            \begin{enumerate}
                \item Implement gravity, so that each second that passes will automatically move the capsule down one
row.
                \item Assuming that gravity has been implemented, have the speed of gravity increase gradually over
time, or after the player completes a certain number of rows
                \item When the player has reached the ”game over” condition, display a Game Over screen in pixels
on the screen. Restart the game if a “retry” option is chosen by the player. Retry should start a
brand new game (no state is retained from previous attempts).
                \item Add sound effects for different conditions like rotating and dropping capsules, removing a row of
squares, for beating a level and the game over condition.
                \item If the player presses the keyboard key p, display a ”Paused” message on screen until they press
p a second time, at which point the original game will resume.
                \item Add levels to the game that trigger after the player eliminates all of the viruses in the current level.
The next level should be more difficult than the previous one (i.e. speed and number of viruses).
Make sure that the increased difficulty is different from the Easy/Medium/Hard difficulty in some
way (the same level can’t count for two features).
                \item Show an outline where the capsule will end up if you drop it.
                \item Have a panel on the side that displays a preview of the next capsule that will appear.
                \item Draw Dr. Mario and the viruses on the side panels, as in Figure 2.1
                \item Assuming you’ve drawn the viruses from the previous feature, have each virus image disappear
as the viruses of that colour are eliminated from the playing field (play the game if you’re unclear
on what this means).
            \end{enumerate}
          \item Hard:
          \begin{enumerate}
              \item Play the Dr. Mario’s theme music in the background while playing the game.
          \end{enumerate}
        \end{enumerate}
    \end{enumerate}

    \item How to view the game:
    % TODO: specify the pixes/unit, width and height of 
    %       your game, etc.  NOTE: list these details in
    %       the header of your breakout.asm file too!
    
    \begin{enumerate}

    \item Unit width in pixels:       2
    \item Unit height in pixels:      2
    \item Display width in pixels:    64
    \item Display height in pixels:   64
    \item Base Address for Display:   0x10008000


    \end{enumerate}

    

\begin{figure}[ht!]
    \centering
    \includegraphics[width=0.7\textwidth]{insturctions.jpeg}
    \caption{How the game looks running on Saturn}
    \label{Instructions}
\end{figure}

\item Game Summary:
\begin{itemize}
    \item This game is our attempt on recreating the famous classic Dr. Mario game, implemented entirely in MIPS assembly.
    \item The player controls falling capsules to align matching colors and eliminate viruses.
    \item Controls are as follows:
    \begin{itemize}
        \item \textbf{W} - Rotate capsule
        \item \textbf{A} - Move capsule left
        \item \textbf{D} - Move capsule right
        \item \textbf{S} - Instantly drop capsule
        \item \textbf{P} - Pause the game
        \item \textbf{Q} - Quit the game
        \item \textbf{R} - Reset the game
    \end{itemize}
    \item We successfully implemented the original "Fever" background music from Dr. Mario to enhance the gameplay experience.
    \item The game features infinite level progression — each new level adds one virus and increases the capsule fall speed (gravity).
    \item The game continues until the player loses, which occurs when capsules block the neck of the medicine bottle (top of the play area).
\end{itemize}

    
\end{enumerate}

\section{Attribution Table}
% TODO: If you worked in partners, tell us who was 
%       responsible for which features. Some reweighting 
%       might be possible in cases where one group member
%       deserves extra credit for the work they put in.

\begin{center}
\renewcommand{\arraystretch}{1.5} 
\begin{tabular}{| p{7cm} | p{7cm} |}
\hline
\textbf{Ahnaf Keenan A. 1010236219} & \textbf{Ibrahim Bilal, 1010437270} \\
\hline
Implemented drawing the medicine bottle & Generated capsules with randomized colors \\
\hline
Implemented left-right movement and gravity & Enhanced movement and added rotation (including instant drop) \\
\hline
Implemented match checking (4+ same-color in a row) & Applied gravity only to unsupported capsules after clearing \\
\hline
Implemented virus drawing with randomized rows and columns spawn points & Improved virus rendering using color variation + normalization for match detection \\
\hline
Implemented win detection (progresses to next level when viruses are cleared) & Implemented Pause and Game Over screens \\
\hline
Added background theme song & Implemented virus display in the sidebar \\
\hline
Added sound effects to user movements (rotate, place down, win) & Implemented capsule preview before it drops \\
\hline
\end{tabular}
\end{center}


\section*{Theme Song Integration}

One of the most challenging and time-consuming parts of the extra features of our project was the creation and integration of the Dr. Mario theme song and sound effects. Rather than using a pre-existing MIDI file or music engine, we chose to manually transcribe the song into MIPS Assembly, note by note, bar by bar.

This required:
\begin{itemize}
    \item Listening closely to the original theme and identifying each note's pitch and rhythm.
    \item Translating each note into a four-field representation: \texttt{<Pitch, Duration, Instrument, Volume>}.
    \item Matching the tempo and structure to fit the pace of the game while maintaining musical accuracy.
    \item Adding bass lines and harmony using a second instrument channel, further enhancing the audio quality.
    \item Ensuring that the music playback was synchronized with the game loop using carefully tuned delays and counters.
\end{itemize}

We implemented a tick-based playback system using a dedicated timer that triggers a note read and playback syscall every fixed number of frames. The song data is stored as a series of \texttt{.word} instructions, and a note index is incremented each time a new note is played. Volume, instrument, and pitch changes are all dynamically handled. Nonetheless, this feature gives the game a polished and nostalgic touch, adding to its charm and authenticity. It's one of the components we're most proud of in this project.



\end{document}